\documentclass[a4paper,11pt]{article} 

\usepackage[english]{babel}
\usepackage[utf8]{inputenc}
\usepackage{amsmath,amssymb}
\usepackage{graphicx}
\usepackage{tikz}
\usetikzlibrary{shapes}
\usepackage{natbib}
\usepackage{hyperref}

\title{Kalman Filter Data Sensitivities \\ An Example of an Adjoint Calculation}

\author{Peter Cotton}

\date{}

\begin{document}
\maketitle

\begin{abstract}
By reformulating the Kalman filter as a least squares problem (in the manner of the Duncan-Horn representation, essentially) and then applying an adjoint calculation, we compute sensitivities of predictions with respect to past observations. 
\end{abstract}

\section{The Kalman Filter}

Consider a sequence of observations $t_1,...,t_k$ at which a latent vector process $x$ is observed indirectly, via an observation equation \begin{equation} y_{t_i} = H_i x_{t_i} + \epsilon_i \end{equation} We assume $\epsilon_i$ is mean zero multivariate gaussian with covariance $R_i$. For brevity we refer to $y_{t_i}$ as $y_i$, $x_{t_i}$ as $x_i$ and so forth. We assume the evolution of $x$ in between the times specified can be written \begin{equation} x_{i+1} = A_i x_i + u_i \end{equation} where $u_i$ are also gaussian. In this linear gaussian system the recursive estimation of $x_t$ is achieved by the well known Kalman filter, and the contemporaneous impact of the next observation $y_{k+1}$ is also (it is merely proportional to the Kalman gain).

But less well appreciated is a related computation, the derivatives of the Kalman filter estimate with respect to a past observation $y_i$. This note establishes how said computation can be achieved by making two observations. The first is a re-representation of the Kalman estimate in the form of a weighted least squares problem (not dissimilar to the Duncan Horn representation). The second observation is that sensitivities of any weighted least squares problem can be computed using an adjoint trick.

\section{The Kalman filter solution as a (particular) least squares problem}

We shall set up a least squares problem involving the current state $x_k$ only, and all the previous observations. We argue that the solution to this problem is identical to the Kalman filter. Since the current estimate $\hat{y}_k$ is a simple linear function of the current state $x_k$, this allows us to compute the derivative of the current estimate with respect to all previous observations.

In the Kalman filter we assume a gaussian prior on the initial state $x_0$. This can introduce annoying special cases in what follows, but we can clean up the notation instead by introducing: \begin{eqnarray}    y_{-1} & = & H_{-1} x_{-1} + \epsilon_{-1} \\ x_0 & = & A_{-1} x_{-1} + u_{-1} \end{eqnarray} provided $H_{-1}$ and $A_{-1}$ are identity matrices, $\epsilon{-1}$ is identically zero, $y_{-1}$ is set equal to the mean of our prior and $u_0$ adopts its covariance. 

With the boundary conditions cleaned up in this fashion we can invert the dynamical equations, assuming only that $A$'s have left inverses $A^{-1}$, as follows: 

\begin{equation} x_j = A^{-1}_{j}\left( x_{j+1} - u_j \right) 
\end{equation} and then re-arrange the observation equations so that the only value of $x_i$ that appears is $x_k$. 
\begin{eqnarray} 
   y_k & = & H_k x_{k} + \epsilon_k \\ y_{k-1} & = & H_{k-1} x_{k-1} + \epsilon_{k-1} \\ & = & H_{k-1} \left( A^{-1}_{k-1}\left( x_{k} - u_{k-1} \right) \right) + \epsilon_{k-1} \\ & = & H_{k-1} A^{-1}_{k-1} x_{k} - H_{k-1} A^{-1}_{k-1} u_{k-1} + \epsilon_{k-1} \\ y_{k-2} & = & H_{k-2} x_{k-2} + \epsilon_{k-2} \\ 
 & = & H_{k-2} \left( A^{-1}_{k-2}\left( x_{k-1} - u_{k-2} \right) \right) + \epsilon_{k-2} \\ & = & H_{k-2} A^{-1}_{k-2} x_{k-1} - H_{k-2} A^{-1}_{k-2} u_{k-2} + \epsilon_{k-2} \\ 
 & = & H_{k-2} A^{-1}_{k-2} \left( A^{-1}_{k-1}\left( x_{k} - u_{k-1} \right) \right) - H_{k-2} A^{-1}_{k-2} u_{k-2} + \epsilon_{k-2} \\     
 & = & H_{k-2} A^{-1}_{k-2} A^{-1}_{k-1} x_{k} - H_{k-2} A^{-1}_{k-2} A^{-1}_{k-1} u_{k-1} - H_{k-2} A^{-1}_{k-2} u_{k-2} + \epsilon_{k-2} \\ & \dots & 
 \end{eqnarray} from which it is apparent that if we write $Y = (y_k, y_{k-1}, y_{k-2},...,y_{-1} ) $ then \begin{equation} Y = G x_{k} + \eta \end{equation} where $G$ is the concatenation of the coefficients of $x_k$ given above and $\eta$ is the gaussian random variable equal to the sum of $u_k$'s and $\epsilon_k$'s (again, with coefficients as above, leading to a non-trivial covariance structure).

\section{The adjoint calculation}

Suppose $x$ solves $Qx = b(y)$. The adjoint trick can be used to compute the derivative of $g(x)$ w.r.t. y. In particular, if $y$ is the observation and $x$ the solution of a generalized least squares problem with error covariance $R$ we can cast it in this form by writing: 
\begin{eqnarray} g(x)& = & H x Q \\ 
                   & = & H^T R^{-1} H \\ 
            b(y) & = & H^T R^{-1} y 
\end{eqnarray} Consider now \begin{equation} f(x,y) = 0 \end{equation} where 
\begin{equation} f(x,y) = Q x - b(y) 
\end{equation} 
We use derivatives of 
\begin{equation} \tilde{g} = g - \lambda^T f(x,y) 
\end{equation} with respect to $y$ as a means of computing derivatives of $g$ with respect to $y$. Note that \begin{equation} \frac{\partial \tilde{g}}{\partial y} = \frac{\partial g}{\partial x}\frac{\partial x}{\partial y} - \lambda^T \left( \frac{ \partial f }{\partial x }\frac{\partial x}{\partial y} + \frac{\partial f}{\partial y} \right) \end{equation} and this will simplify if we choose $\lambda$ judiciously as a solution of \begin{equation} \frac{\partial g}{\partial x } = \lambda^T \frac{\partial f}{\partial x} \end{equation} which we call the adjoint equation. For then 
\begin{eqnarray} \frac{\partial \tilde{g}}{\partial y} 
      & = & \frac{\partial g}{\partial x}\frac{\partial x}{\partial y} - \lambda^T \left( \frac{ \partial f }{\partial x }\frac{\partial x}{\partial y} + \frac{\partial f}{\partial y} \right) \\
      & = & -\lambda^T \frac{\partial f}{\partial y} \\
      & = & \lambda^T \frac{\partial b}{\partial y} \end{eqnarray}
Now specializing to \begin{equation} g(x) = H x \end{equation} and $b(y)$ as above we can solve for this convenient choice of $\lambda$ by writing 
\begin{eqnarray} H & = & \frac{\partial g}{\partial x} \\
     & = & \lambda^T \frac{\partial f}{\partial x} \\
     & = & \lambda^T Q \\ 
     & = & \lambda^T H^T R^{-1} H 
     \end{eqnarray} where the second equality is the adjoint equation. It should be clear from this how to compute derivatives of $\tilde{g}$ with respect to $y$, and thereby compute derivatives of $g$ with respect to $y$.

\bibliographystyle{unsrt}

\end{document}